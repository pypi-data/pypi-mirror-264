\chapter{Knowledge graph}

  \code{djehuty} processes its information using the Resource Description
  Framework \citep{lassila-99-rdf}.  This chapter describes the parts that
  make up the data model of \t{djehuty}.

\section{Use of vocabularies}

  Throughout this chapter, abbreviated references to ontologies are used.
  Table \ref{table:vocabularies} lists these abbreviations.

  \hypersetup{urlcolor=black}
  \begin{table}[H]
    \begin{tabularx}{\textwidth}{*{1}{!{\VRule[-1pt]}l}!{\VRule[-1pt]}X}
      \headrow
      \b{Abbreviation} & \b{Ontology URI}\\
      \evenrow
      \t{djht}         & \dhref{https://ontologies.data.4tu.nl/djehuty/\djehutyversion/}\\
      \oddrow
      \t{rdf}          & \dhref{http://www.w3.org/1999/02/22-rdf-syntax-ns\#}\\
      \evenrow
      \t{rdfs}         & \dhref{http://www.w3.org/2000/01/rdf-schema\#}\\
      \oddrow
      \t{xsd}          & \dhref{http://www.w3.org/2001/XMLSchema\#}\\
    \end{tabularx}
    \caption{\small Lookup table for vocabulary URIs and their abbreviations.}
    \label{table:vocabularies}
  \end{table}
  \hypersetup{urlcolor=LinkGray}

\section{Notational shortcuts}

  In addition to abbreviating ontologies with their prefix we use another
  notational shortcut.  To effectively communicate the structure of the RDF
  graph used by \t{djehuty} we introduce a couple of shorthand notations.

\subsection{Notation for typed triples}

  When the \code{object} in a triple is \i{typed}, we introduce the shorthand
  to only show the type, rather than the actual value of the \code{object}.
  Figure \ref{fig:typed-notation} displays this for URIs, and figure
  \ref{fig:typed-literals-notation} displays this for literals.

  \includefigure{typed-notation}{Shorthand notation for triples with an
    \code{rdf:type} which features a hollow predicate arrow and a colored
    type specifier with rounded corners.}

  Literals are depicted by rectangles (with sharp edges) in contrast to URIs
  which are depicted as rectangles with rounded edges.

  \includefigure{typed-literals-notation}{Shorthand notation for triples with
    a literal, which features a hollow predicate arrow and a colored
    rectangular type specifier.}

  When the subject of a triple is the shorthand type, assume the subject is not
  the type itself but the subject which has that type.

\subsection{Notation for \code{rdf:List}}

  To preserve the order in which lists were formed, the data model makes use
  of \code{rdf:List} with numeric indexes.  This pattern will be abbreviated
  in the remainder of the figures as displayed in figure
  \ref{fig:rdf-list-abbrev}.

  \includefigure{rdf-list-abbrev}{Shorthand notation for \code{rdf:List}
    with numeric indexes, which features a hollow double-arrow.  Lists have
    arbitrary lengths, and the numeric indexes use 1-based indexing.}

  The hollow double-arrow depicts the use of an \code{rdf:List} with numeric
  indexes.

\section{Datasets}

  Datasets play a central role in the repository system because every
  other type links in one way or another to it.  The user submits
  files along with data about those bytes as a single record which we
  call a \code{djht:Dataset}.  Figure \ref{fig:dataset} shows how the
  remainder of types in this chapter relate to a \code{djht:Dataset}.

  \includefigure{dataset}{The RDF pattern for a \code{djht:Dataset}.
    For a full overview of \code{djht:Dataset} properties, use the exploratory
    from the administration panel.}

  Datasets are versioned records.  The data and metadata between versions
  can differ, except all versions of a dataset share an identifier.  We use
  \code{djht:DatasetContainer} to describe the version-unspecific properties
  of a set of versioned datasets.

  \includefigure{dataset-container}{The RDF pattern for a
    \code{djht:DatasetContainer}.  All versions of a dataset share a
    \code{djht:dataset\_id} and a UUID in the container URI.}

  The data model follows a natural expression of published versions as a
  linked list.  Figure \ref{fig:dataset-container} further reveals that
  the \i{view}, \i{download}, \i{share} and \i{citation} counts are stored
  in a version-unspecific way.

\section{Accounts}

  \t{djehuty} uses an external identity provider, but stores an e-mail address,
  full name, and preferences for categories.

  \includefigure{account}{The RDF pattern for an \code{djht:Account}.}

\section{Funding}

  When the \code{djht:Dataset} originated out of a funded project, the funders
  can be listed using \code{djht:Funding}.  Figure \ref{fig:funding} displays
  the details for this structure.

  \includefigure{funding}{The RDF pattern for a \code{djht:Funding}.}

